\section{Decentralized Finance (DeFi)}

DeFi, blockchain teknolojisi üzerine kurulu, aracıları ve merkezi yapıları ortadan kaldırarak finansal hizmetler sunan bir ekosistemdir. Geleneksel finans sisteminin sunduğu hizmetleri şeffaf, güvenilir ve merkeziyetsiz bir ortamda gerçekleştirmeyi hedefler. Finansal sistemin erişilebilirliğini artırmayı, aracılara olan bağımlılığı azaltmayı ve kullanıcılara finansal varlıkları üzerinde tam kontrol sağlamayı amaçlar. İşlemler merkezi yapılar yerine akıllı sözleşmeler ile gerçekleştirilir. Kullanıcılar, teminat göstererek kredi alabilir veya varlıklarını likidite sağlayıcı havuzlara ekleyerek faiz kazanabilir. DeFi ekosisteminde, işlem kolaylığı ve istikrar sağlamak için kullanılan sabit fiyatlı Stablecoin adı verilen kripto varlıklar kullanılırs. Kullanıcılar, DEX (merkezi olmayan borsa) platformları üzerinde kripto varlıklarını doğrudan birbirleriyle değiştirebilir.

\subsection{Staking}

Staking, bir blockchain ağında kripto varlıkların belirli bir süre boyunca kilitlenerek, ağı güvence altına alınmasını ve işlem doğrulanmasına katkıda bulunmaya dayalı bir mekanizmadır. Bu işlem karşılığında kullanıcılar ödül alır. PoS tabanlı sistemlerde kullanılır. Stake edilen varlıklar, platformun likiditesini artırır ve hizmetlerin sürdürülebilirliğini sağlar.

\subsection{Likidite}

Likidite, bir platformun kullanıcılarına hizmet verebilmesi için yeterli miktarda yoken veya varlık rezervine sahip olmasını ifade eder. Kullanıcıların ticaret, borç alma/verme gibi işlemleri sorunsuzca gerçekleştirmesini sağlar. Düşük likidite, fiyat oynaklığına ve işlem maliyetlerinin artmasına neden olabilir. Likidite,  kullanıcıların platformlara token yatırmasıyla sağlanır. Sağlanan likidite, merkezi olmayan borsalarda (DEX) veya borç protokollerinde kullanılabilir.

\subsection{Likidite Havuzları}

Likidite Havuzları, kullanıcıların tokenlerini bir araya getirdiği ve diğer kullanıcıların bu havuzdan işlem yapmasını sağlayan bir sistemdir. DEX platformlarında ticaret yapılmasını sağlar. Borç alma ve verme işlemlerine kaynak oluşturur. Platforma likidite sağlayarak işlem hacmini artırır.

\begin{enumerate}
    \item Kullanıcılar, eşit miktarda iki farklı tokenı likidite havuzuna yatırır.
    \item Diğer kullanıcılar, havuzdan token alıp satarak ticaret yapar. Alım-satım işlemleri sırasında likidite sağlayıcıları küçük bir işlem ücreti kazanır.
    \item Likidite sağlayıcıları, yatırdıkları varlıkları ve kazandıkları işlem ücretlerini, "LP token" adı verilen bir temsili token ile yönetir.
\end{enumerate}

\subsection{Yield Farming}

Yield farming, kullanıcıların DeFi protokollerine varlıklarını ödünç vererek veya likidite sağlayarak getiri elde etmesini sağlayan bir yatırım stratejisidir. Kullanıcıların pasif gelir elde etmesini sağlar. Platformların likidite sorunlarını çözmesine yardımcı olur.

\newpage