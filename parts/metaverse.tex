\section{Metaverse}

Metaverse, VR teknolojileri kullanılarak inşa edilmiş, kullanıcıların dijital kimliklerle varlık gösterebildiği, sosyal etkileşimde bulunabildiği, ticaret yapabildiği ve oyun oynayabildiği bir dijital evrendir. Birden fazla sanal dünyanın birleşiminden oluşur ve bu dünyalar, gerçek dünya aktivitelerini dijital ortama taşır. 

\begin{itemize}
    \item \textbf{Sanallık}: Gerçek dünyadan bağımsız, dijital bir ortama sahiptir.
    \item \textbf{Etkileşim}: Kullanıcılar, diğer varlıklar ile dinamik bir şekilde etkileşimde bulunabilir.
    \item \textbf{Süreklilik}: Sürekli açık, gelişen ve kullanıcıların etkileşimlerinin kalıcı olduğu bir ortama sahiptir.
\end{itemize}

Metaverse, kullanım alanları:

\begin{itemize}
    \item Uzaktan eğitim, sanal ofisler ve iş toplantıları için kullanılabilir.
    \item Sanal konserler, oyunlar, etkinlikler düzenlenebilir.
    \item Kullanıcılar dijital varlıklar satın alabilir, satabilir veya kiralayabilir.
    \item Sanal arazi satın alma, iş kurma gibi faaliyetler gerçekleştirilebilir.
\end{itemize}

\newpage