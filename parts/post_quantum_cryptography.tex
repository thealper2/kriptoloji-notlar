\section{Post-Quantum Cryptography}

Post-Kuantum Kriptografi, kuantum bilgisayarların geleneksel kriptografi yöntemlerini tehdit etmesinden dolayı ortaya çıkmıştır. Geleneksel şifreleme yöntemleri büyük ölçüde matematiksel problemlerin çözüm zorluğuna dayanır. Ancak kuantum bilgisayarlar, güçlü algoritmalar sayesinde bu matematiksel problemleri etkili bir şekilde çözebilir. Post-Kuantum Kriptografi, kuantum bilgisayarların bu potansiyel tehditlerine karşı dayanıklı algoritmalar sunmayı amaçlar. Bu algoritmalar, kuan tum bilgisayarların güçlü işlem kapasitesine dayanacak şekilde tasarlanmıştır. Sadece kuantum saldırıları değil, aynı zamanda klasik bilgisayarların saldırılarına karşı da dayanıklıdır. Henüz tam olarak benimsenmiş standartlar yoktur ve standartlaştırma süreci devam etmektedir. PQC, problem kategorileri;

\begin{itemize}
    \item \textbf{Izgara (Lattice)}: Çok boyutlu geometrik ızgaralarda kısa vektörlerin bulunması zorluğuna dayanır. Kullanılan algoritmalar: Learning with Errors (LWE), NTRU.
    \item \textbf{Kodlama (Code)}: Hata düzeltme kodlarının çözümündeki zorluğa dayanır. Kullanılan algoritmalar: McEliece.
    \item \textbf{Çok Değişkenli Polinomlar (Multivariate Polynomials)}: Çok değişkenli polinomların sıfırlarının bulunmasındaki zorluğa dayanır. Kullanılan algoritmalar: Rainbow, UOV (Unbalanced Oil and Vinegar).
    \item \textbf{Hash}: Güvenli hash fonksiyonlarına dayanır. Kullanılan algoritmalar: SPHINCS+.
    \item \textbf{Diğer}: Farklı matematiksel zorluklardan yararlanır. Kullanılan algoritmalar: Supersingular Elliptic Curve Isogeny (SIEC).
\end{itemize}

\newpage