\section{Tarihteki Şifreleme Yöntemleri}

\subsection{Polybius Cipher}

Polybius Cipher, eski Yunan filozofu Polybius tarafından önerilmiştir. Bu yöntem, bir tablo kullanılarak hashlerin şifrelenmesi prensibine dayanır. Her harfe tablo üzerinde bir koordinat atanır. İngiliz alfabesinin 26 harfinden biri eksiltilir ve 5x5 bir kare oluşturulur.

\begin{table}[ht]
\centering
\resizebox{0.5\textwidth}{!}{%
\begin{tabular}{|l|l|l|l|l|l|}
\hline
  & 1 & 2 & 3 & 4 & 5 \\ \hline
1 & A & B & C & D & E \\ \hline
2 & F & G & H & I/J & K \\ \hline
3 & L & M & N & O & P \\ \hline
4 & Q & R & S & T & U \\ \hline
5 & V & W & X & Y & Z \\ \hline
\end{tabular}%
}
\end{table}

Şifreleme sırasında her harf, tablodaki satır ve sütun numarası ile temsil edilir. Şifre çözme sırasında bu sayılar tekrar harflere dönüştürülür. Örneğin "ALPER" kelimesinin şifrelenmiş hali "1131351542" dir.

\subsubsection{Encryption}

\begin{enumerate}
    \item Bir Polybius tablosu oluşturulur.
    \item Şifrelenecek metnin her harfi için koordinat bulunur.
    \item Bu koordinatlar birleştirilerek şifreli metin elde edilir.
\end{enumerate}

\subsubsection{Decryption}

\begin{enumerate}
    \item Şifrelenmiş metin iki rakamlı gruplara ayrılır.
    \item Her grubun satır ve sütun değerine karşılık gelen harf bulunur.
    \item Harfler birleştirilerek orijinal metin elde edilir.
\end{enumerate}

\subsubsection{Python Kodu}

\begin{lstlisting}[language=Python]
def create_polybius_square():
    alphabet = "ABCDEFGHIKLMNOPQRSTUVWXYZ"
    square = {}
    index = 0
    for row in range(1, 6):
        for col in range(1, 6):
            square[alphabet[index]] = (row, col)
            index += 1

    coordinates = {v: k for k, v in square.items()}
    return square, coordinates

def encrypt_polybius(text):
    square, _ = create_polybius_square()
    text = text.upper().replace("J", "I")
    encrypted = ""
    for char in text:
        if char.isalpha():
            row, col = square[char]
            encrypted += f"{row}{col}"
        else:
            encrypted += char

    return encrypted

def decrypt_polybius(text):
    _, coordinates = create_polybius_square()
    decrypted = ""
    i = 0
    while i < len(text):
        if text[i].isdigit() and i + 1 < len(text) and text[i + 1].isdigit():
            row = int(text[i])
            col = int(text[i + 1])
            decrypted += coordinates[(row, col)]
            i += 2
        else:
            decrypted += text[i]
            i += 1

    return decrypted

plaintext = "Hello World"
ciphertext = encrypt_polybius(plaintext)
decrypted_text = decrypt_polybius(ciphertext)
print("Ciphertext:", ciphertext)
print("Decrypted:", decrypted_text)
\end{lstlisting}

\newpage

\subsection{Caesar Cipher}

Caesar Cipher, Julius Caesar tarafından kullanılan bir yer değiştirme (substitution) şifreleme yöntemidir. Bu yöntemde, alfabenin harfleri sabit bir sayı kadar kaydırılarak şifrelenir. Şifreleme ve deşifreleme işlemi, kaydırma miktarı (shift değeri) üzerinden yapılır. Eğer shift değeri 3 ise, A harfi D harfi olarak şifrelenir. Bu mantık şifreleme sırasında tüm harflere uygulanır. Deşifreleme ise tam tersine kaydırarak yapılır. Caesar algoritması oldukça güvensizdir çünkü brute force saldırıları ile tüm denemeler yapılarak düz metin kolayca elde edilir.

\subsubsection{Encryption}

\begin{enumerate}
    \item Shift değeri belirlenir.
    \item Şifrelenecek metindeki her harf için; harfin alfabe sırasındaki yerine shift değeri eklenir.
    \item Eğer alfabe sonuna ulaşılırsa başa dönülür.
\end{enumerate}

\subsubsection{Decryption}

\begin{enumerate}
    \item Şifrelenmiş metindeki her harf için; harfin alfabe sıraısndaki yerinden shift değeri çıkarılır.
    \item Eğer alfabe başından önceye gidilerse sona dönülür.
\end{enumerate}

Örneğin "ALPER" kelimesinin 2 kelimelik shiftler ile şifrelenmiş hali "CNRGT" dir.

\subsubsection{Python Kodu}

\begin{lstlisting}[language=Python]
def caesar_encrypt(plainttext, shift=3):
    encrypted = ""
    for char in plainttext:
        if char.isalpha():
            base = ord('A') if char.isupper() else ord('a')
            encrypted += chr((ord(char) - base + shift) % 26 + base)
        else:
            encrypted += char

    return encrypted

def caesar_decrypt(ciphertext, shift=3):
    decrypted = ""
    for char in ciphertext:
        if char.isalpha():
            base = ord('A') if char.isupper() else ord('a')
            decrypted += chr((ord(char) - base - shift) % 26 + base)
        else:
            decrypted += char

    return decrypted

plaintext = "Hello World"
ciphertext = caesar_encrypt(plaintext, shift=3)
decrypted_text = caesar_decrypt(ciphertext, shift=3)
print("Ciphertext:", ciphertext)
print("Decrypted:",decrypted_text)
\end{lstlisting}

\newpage

\subsection{Affine Cipher}

Affine Cipher, bir yer değiştirme şifreleme yöntemidir. Her harf, bir matematiksel formül kullanılarak şifrelenir. Şifreleme işlemi, modüler aritmetiğe dayanır ve iki anahtar kullanır: $a$ (çarpan) ve $b$ (toplama sabiti). Bu yöntem, monoalfabetik bir şifreleme türüdür. Şifreleme için:

\[ E(x) = (\alpha \cdot x + b) \text{ mod } 26 \]

Burada, $x$, harfin alfabe üzerindeki sırasını (0-25 arasında); $\alpha$, çarpanı (mod 26 ile asal olmalıdır); $b$, toplama sabitini; $\text{ mod } 26$, alfabe boyutunu temsil eder. Deşifreleme için:

\[ D(y) = \alpha^{-1} \cdot (y - b) \text{ mod } 26 \]

Burada, $y$, şifreli harfin alfabe üzerindeki sırasını, $\alpha^{-1}$, $\alpha$'nın mod 26'ya göre ters çarpanını temsil eder.

Örneğin, "ALPER" kelimesini Affine Cipher ile şifreleyelim. $\alpha = 5$ ve $b = 8$ olsun.

\begin{table}[ht]
\centering
\resizebox{0.8\textwidth}{!}{%
\begin{tabular}{|l|l|l|l|l|l|l|l|l|l|l|l|l|}
\hline
A & B & C & D & E & F & G & H & I & J & K & L & M \\ \hline
0 & 1 & 2 & 3 & 4 & 5 & 6 & 7 & 8 & 9 & 10 & 11 & 12 \\ \hline
N & O & P & Q & R & S & T & U & V & W & X & Y & Z \\ \hline
13 & 14 & 15 & 16 & 17 & 18 & 19 & 20 & 21 & 22 & 23 & 24 & 25 \\ \hline
\end{tabular}%
}
\end{table}

Tabloya göre şifrelenmiş mesaj şu şekildedir:

\begin{itemize}
    \item A harfi için $y = (5 \cdot 0 + 8) \text{ mod } 26 = \text{I}$
    \item L harfi için $y = (5 \cdot 9 + 8) \text{ mod } 26 = \text{L}$
    \item P harfi için $y = (5 \cdot 15 + 8) \text{ mod } 26 = \text{F}$
    \item E harfi için $y = (5 \cdot 4 + 8) \text{ mod } 26 = \text{C}$
    \item R harfi için $y = (5 \cdot 16 + 8) \text{ mod } 26 = \text{P}$
\end{itemize}

"ALPER" kelimesinin Affine Cipher ile şifrelenmiş hali "ILFCP" dir. Şimdi ise bu şifreyi çözelim. Her bir harf için:

\begin{itemize}
    \item $1 = \alpha \cdot \alpha^{-1} \text{ mod } 26$, buradan $\alpha_{\text{inv}} = 21$ olur.
    \item "I" karakteri için, tabloda değeri 8, $D(x) = 21 \cdot (8 - 8) \text{ mod } 26 = 0$ olur. 0'ın tablodaki değeri "A"'dır.
    \item "L" karakteri için, tabloda değeri 11, $D(x) = 21 \cdot (11 - 8) \text{ mod } 26 = 11$ olur. 11'in tablodaki değeri "L"'dır.
    \item "F" karakteri için, tabloda değeri 5, $D(x) = 21 \cdot (5 - 8) \text{ mod } 26 = 15$ olur. 15'in tablodaki değeri "P"'dır.
    \item "C" karakteri için, tabloda değeri 2, $D(x) = 21 \cdot (2 - 8) \text{ mod } 26 = 4$ olur. 4'ün tablodaki değeri "E"'dır.
    \item "P" karakteri için, tabloda değeri 15, $D(x) = 21 \cdot (15 - 8) \text{ mod } 26 = 17$ olur. 17'nin tablodaki değeri "R"'dır.
\end{itemize}

Böylece, şifreli metin "ILFCP" den tekrar "ALPER" kelimesini elde etmiş olduk. 

\subsubsection{Python Kodu}

\begin{lstlisting}[language=Python]
import math

def mod_inverse(a, m):
    for x in range(1, m):
        if (a * x) % m == 1:
            return x

    return None

def affine_encrypt(plaintext, a=5, b=8):
    if math.gcd(a, 26) != 1:
        raise ValueError("")

    encrypted = ""
    for char in plaintext.upper():
        if char.isalpha():
            x = ord(char) - ord("A")
            encrypted += chr(((a * x + b) % 26) + ord("A"))
        else:
            encrypted += char

    return encrypted

def affine_decrypt(ciphertext, a=5, b=8):
    if math.gcd(a, 26) != 1:
        raise ValueError("")

    a_inv = mod_inverse(a, 26)
    if a_inv is None:
        raise ValueError

    decrypted = ""
    for char in ciphertext.upper():
        if char.isalpha():
            y = ord(char) - ord("A")
            decrypted += chr(((a_inv * (y - b)) % 26) + ord("A"))

        else:
            decrypted += char

    return decrypted

plaintext = "Hello World"
ciphertext = affine_encrypt(plaintext, a=5, b=8)
decrypted_text = affine_decrypt(ciphertext, a=5, b=8)
print("Ciphertext:", ciphertext)
print("Decrypted:", decrypted_text)
\end{lstlisting}

\newpage

\subsection{Vigenere Cipher}

Vigenere Cipher, çok alfabeli yer değiştirme şifreleme (polyalphabetic substitution cipher) yöntemidir. 1553 yılında Giovan Battista Bellaso tarafından tanıtılmıştır ve 1586'da Fransız diplomat Blaise de Vigenere tarafından geliştirilmiştir. Şifreleme ve deşifreleme işlemi, bir anahtar kelime kullanılarak gerçekleştirilir. Anahtar kelime, şifrelenecek metnin uzunluğuna kadar tekrar eder. Bu yöntem, basit yer değiştirme şifrelemelerine göre daha güvenlidir, çünkü farklı alfabeler arasında geçiş yaparak daha karmaşık bir şifreleme sağlar. Şifreleme için, her harf bir anahtar kelime yardımıyla bir alfabe içinde kaydırılır. Caesar algoritmasının geliştirilmiş bir halidir.

\[ C_i = (P_i + K_i) \text{ mod } 26 \]

Burada, $P_i$, düz metindeki harfin alfabe sırası (0-25 arasında); $K_i$, anahtar keliminin ilgili harfinin alfabe sırası (0-25 arasında); $C_i$, şifrelenmiş metindeki harfin alfabe sırasıdır. Deşifreleme işleminde, şifreleme işlerminin tersi yapılır. 

\[ P_i = (C_i - K_i) \text{ mod } 26 \]

Örneğin "ANAHTAR" kelimesini kullanarak "ALPERKARACA" ismini şifreleyelim. 

\begin{table}[ht]
\centering
\resizebox{0.8\textwidth}{!}{%
\begin{tabular}{|l|l|l|l|l|l|l|l|l|l|l|l|l|}
\hline
A & B & C & D & E & F & G & H & I & J & K & L & M \\ \hline
0 & 1 & 2 & 3 & 4 & 5 & 6 & 7 & 8 & 9 & 10 & 11 & 12 \\ \hline
N & O & P & Q & R & S & T & U & V & W & X & Y & Z \\ \hline
13 & 14 & 15 & 16 & 17 & 18 & 19 & 20 & 21 & 22 & 23 & 24 & 25 \\ \hline
\end{tabular}%
}
\end{table}

Tabloya göre şifrelenmiş mesaj şu şekildedir:

\begin{table}[ht]
\resizebox{\textwidth}{!}{%
\begin{tabular}{|l|l|l|l|l|l|}
\hline
Anahtar Harf & Mesaj Harf & Index Toplamı & Mod & Şifreli Mesaj Index & Şifreli Harf \\ \hline
A & A & $0 + 0 = 0$ & $0 \text{ mod } 26 = 0$ & 0 & A \\ \hline
N & L & $13 + 11 = 24$ & $24 \text{ mod } 26 = 24$ & 24 & Y \\ \hline 
A & P & $0 + 15 = 15$ & $15 \text{ mod } 26 = 15$ & 15 & P \\ \hline
H & R & $7 + 4 = 11$ & $11 \text{ mod } 26 = 11$ & 11 & L \\ \hline
T & T & $19 + 17 = 36$ & $36 \text{ mod } 26 = 10$ & 10 & K \\ \hline
A & K & $0 + 10 = 10$ & $10 \text{ mod } 26 = 10$ & 10 & K \\ \hline
R & A & $17 + 0 = 17$ & $17 \text{ mod } 26 = 17$ & 17 & R \\ \hline
A & R & $0 + 17 = 17$ & $17 \text{ mod } 26 = 17$ & 17 & R \\ \hline
N & A & $13 + 0 = 13$ & $13 \text{ mod } 26 = 13$ & 13 & N \\ \hline
A & C & $0 + 2 = 2$ & $2 \text{ mod } 26 = 2$ & 2 & C \\ \hline
H & A & $7 + 0 = 7$ & $7 \text{ mod } 26 = 7$ & 7 & H \\ \hline
\end{tabular}%
}
\end{table}

\newpage

Buradan "ANAHTAR" kelimesi anahtarı ile "ALPERKARACA" mesajını şifrelenmiş hali "AYPLKKRRNCH" geliyor. Şimdi ise bunu aynı anahtar ile geri çözelim. Şifre çözme işlemi için şifreli mesaj harflerinin değerlerinden, anahtar kelimenin harflerinin değerleri çıkarılır. Eğer sonuç sıfırdan küçükse üzerine mod yani 26 eklenir.

\begin{table}[ht]
\resizebox{\textwidth}{!}{%
\begin{tabular}{|l|l|l|l|l|l|}
\hline
Şifreli Mesaj Harf & Anahtar Harf & Index Toplamı & Mod & Mesaj Index & Mesaj Harf \\ \hline
A & A & $0 - 0 = 0$ & $0$ & 0 & A \\ \hline
Y & N & $24 - 13 = 11$ & $11$ & 11 & L \\ \hline 
P & A & $15 - 0 = 15$ & $15$ & 15 & P \\ \hline
L & H & $11 - 7 = 4$ & $4$ & 4 & E \\ \hline
K & T & $10 - 19 = -9$ & $-9 + 26 = 17$ & 17 & R \\ \hline
K & A & $10 - 0 = 10$ & $10$ & 10 & K \\ \hline
R & R & $17 - 17 = 0$ & $0$ & 0 & A \\ \hline
R & A & $17 - 0 = 17$ & $17$ & 17 & R \\ \hline
N & N & $13 - 13 = 0$ & $0$ & 0 & A \\ \hline
C & A & $2 - 0 = 2$ & $2$ & 2 & C \\ \hline
H & H & $7 - 7 = 0$ & $0$ & 0 & A \\ \hline
\end{tabular}%
}
\end{table}

Görüldüğü gibi tekrardan "ALPERKARACA" kelimesini elde ettik.

\subsubsection{Python Kodu}

\begin{lstlisting}[language=Python]
def vigenere_encrypt(plaintext, key):
    key = key.upper()
    encrypted = ""
    key_index = 0
    for char in plaintext.upper():
        if char.isalpha():
            p = ord(char) - ord("A")
            k = ord(key[key_index]) - ord("A")
            encrypted += chr((p + k) % 26 + ord("A"))
            key_index = (key_index + 1) % len(key)
        else:
            encrypted += char

    return encrypted

def vigenere_decrypt(ciphertext, key):
    key = key.upper()
    decrypted = ""
    key_index = 0
    for char in ciphertext.upper():
        if char.isalpha():
            c = ord(char) - ord('A')
            k = ord(key[key_index]) - ord('A')
            decrypted += chr((c - k + 26) % 26 + ord('A'))
            key_index = (key_index + 1) % len(key)
        else:
            decrypted += char 
    return decrypted

plaintext = "Hello World"
key = "secretkey"
ciphertext = vigenere_encrypt(plaintext, key)
decrypted_text = vigenere_decrypt(ciphertext, key)
print("Ciphertext:", ciphertext)
print("Decrypted:", decrypted_text)
\end{lstlisting}

\newpage

\subsection{Playfair Cipher}

1854 yılında Charles Wheatstone tarafından geliştirilmiştir. Carles Wheatsone tarafından geliştirilmiş olmasına rağmen adını bu algoritmanın kullanılmasını savunan Lord Playfair'den alır. Çift harfli (digraph) yer değiştirme şifreleme yöntemidir. Şifreleme işlemi 5x5 harf matrisinin kullanımıyla gerçekleştirilir. Bu yöntem, metni harf çiftleri (digraph) halinde işler ve bu çiftlerin matris üzerindeki pozisyonlarına göre şifreler. İlk yıllarda yöntemi karmaşık bulan İngiliz Dışişleri Bakanlığı daha sonra 1. ve 2. Dünya Savaşlarında bu yöntemi kullanmıştır. 2. Dünya Savaşında Avustral ve Yeni Zelanda da kullanmıştır.  

\subsubsection{Encryption}

\begin{enumerate}
    \item Anahtar kelime alınır ve tekrar eden harfler çıkarılarak matrisin ilk satırlarına yazılır. Matris, geriye kalan alfabe harfleriyle doldurulur.
    \item Düz metin iki harfli gruplara bölünür. Aynı harfler bir çift içinde yer alırsa, araya "X" eklenir. Tek harfli kalan metinler için sona bir "X" eklenir.
    \item Her harf çifti için matris üzerindeki konumlarına göre şifrelenir:
    \begin{itemize}
        \item \textbf{Aynı Satır}: Harfler sağa doğru kaydırılır.
        \item \textbf{Aynı Sütun}: Harfler aşağıya doğru kaydırılır.
        \item \textbf{Farklı Satır ve Sütun}: Harfler dikdörtgenin karşı köşeleriyle değiştirilir.
    \end{itemize}
\end{enumerate}

Örneğin, "MONARCHY" anahtarı ile "ALPERKARACAD" kelimesini şifreleyelim. "MONARCHY" anahtarında tekrar eden bir harf olmadığı için anahtar aynı şekilde matrise eklenir. Matris:

\begin{table}[ht]
\centering
\resizebox{0.5\textwidth}{!}{%
\begin{tabular}{|l|l|l|l|l|}
\hline
M & O & N & A & R \\ \hline
C & H & Y & B & D \\ \hline
E & F & G & I/J & K \\ \hline
L & P & Q & S & T \\ \hline
U & V & W & X & Z \\ \hline
\end{tabular}%
}
\end{table}

Şimdi "ALPERKARACA" kelimesini ikili gruplara ayıralım: "AL PE RK AR AC AD". Tabloya göre:

\begin{itemize}
    \item Köşeleri A ve L olan karenin diğer köşeleri: MS.
    \item Köşeleri P ve E olan karenin diğer köşeleri: LF.
    \item R ve K aynı sütunda olduğu için birer altlarındaki değerler: DT.
    \item A ve R aynı satırda olduğu için birer sağındakiler: RM.
    \item Köşeleri A ve C olan karenin diğer köşeleri: MB.
    \item Köşeleri A ve D olan karenin diğer köşeleri: RB.
\end{itemize}

"MONARCHY" anahtarı ile "ALPERKARACAD" kelimesinin şifrelenmiş hali "MSLFDTRMMBRB" dir.

\newpage

\subsection{Bifid Cipher}

1901 yılında Fransız kriptograf Felix Delastelle tarafından geliştirilmiştir. Felix Delastelle daha sonra Trifid ve Four-square şifreleme yöntemlerini de geliştirmiştir. Bu yöntem, Polybios karesi kullanarak hem yer değiştirme hem de transpozisyon şifreleme tekniklerini birleştirir. Amaç, bir mesajın harflerini hem satır hem de sütun koordinatları üzerinden şifrelemektir. 

\subsubsection{Encryption}

\begin{enumerate}
    \item Polybios karesini oluşturmak için alfabenin harfleri, bir 5x5 matris içinde yerleştirilir. Anahtar kelime, matrisin doldurulmasında ilk sırayı alır ve ardından geri kalan alfabe harfleri eklenir.
    \item Düz metin (plaintext) yalnızca alfabe harflerinden oluşmalıdır ve büyük harflerle yazılmalıdır. 
    \item Her harf için matrisin satır ve sütun koordinatları alınır ve bir listeye yazılır.
    \item Satır ve sütun koordinatları birleştirilerek yeni bir sıralama oluşturulur.
    \item Oluşan yeni sıralama, her çift sayı için matristen harf seçilerek şifreli metin (ciphertext) elde edilir.
\end{enumerate}

Örneğin, "ANAHTAR" kelimesi anahtarı ile "ALPERKARACA" kelimesini şifreleyelim. Bifid matrisi:

\begin{table}[ht]
\centering
\resizebox{0.5\textwidth}{!}{%
\begin{tabular}{|l|l|l|l|l|l|}
\hline
  & 1 & 2 & 3 & 4 & 5 \\ \hline
1 & A & N & H & T & R \\ \hline
2 & B & C & D & E & F \\ \hline
3 & G & I/J & K & L & M \\ \hline
4 & O & P & Q & S & U \\ \hline
5 & V & W & X & Y & Z \\ \hline
\end{tabular}%
}
\end{table}

\newpage

Tabloya göre "ALPERKARACA" kelimesinin şifrelenmiş hali:

\begin{table}[ht]
\centering
\resizebox{0.8\textwidth}{!}{%
\begin{tabular}{|l|l|l|l|l|l|l|l|l|l|l|l|}
\hline
Harf & A & L & P & E & R & K & A & R & A & C & A \\ \hline
Satır & 1 & 3 & 4 & 2 & 1 & 3 & 1 & 1 & 1 & 2 & 1 \\ \hline
Sütun & 1 & 4 & 2 & 4 & 5 & 3 & 1 & 5 & 1 & 2 & 1 \\ \hline
\end{tabular}%
}
\end{table}

Şimdi bu tablodaki değerleri soldan sağa doğru ikişerli şekilde okuyarak Bifid tablosundaki değeriyle mesajı şifreleyelim:

\begin{itemize}
    \item 13, 1. satır 3. sütun: H
    \item 42, 4. satır 2. sütun: P
    \item 13, 1. satır 3. sütun: H
    \item 11, 1. satır 1. sütun: A
    \item 12, 1. satır 2. sütun: N
    \item 11, 1. satır 1. sütun: A
    \item 42, 4. satır 2: sütun: P
    \item 45, 4. satır 5. sütun: U
    \item 31, 3. satır 1. sütun: G
    \item 51, 5. satır 1. sütun: V
    \item 21, 2. satır 1. sütun: B
\end{itemize}

"ANAHTAR" anahtarı ile "ALPERKARACA" mesajının şifrelenmiş hali "HPHANAPUGVB" dır.

\subsubsection{Python Kodu}

\begin{lstlisting}[language=Python]
def create_polybius_square(key):
    key = key.upper().replace("J", "I")
    alphabet = "ABCDEFGHIKLMNOPQRSTUVWXYZ"
    key_square = []
    used_chars = set()
    for char in key:
        if char not in used_chars and char in alphabet:
            key_square.append(char)
            used_chars.add(char)

    for char in alphabet:
        if char not in used_chars:
            key_square.append(char)

    square = [key_square[i:i + 5] for i in range(0, 25, 5)]
    positions = {char: (i // 5, i % 5) for i, char in enumerate(key_square)}
    return square, positions

def bifid_encrypt(plaintext, key):
    square, positions = create_polybius_square(key)
    plaintext = plaintext.upper().replace('J', 'I')
    row_coords = []
    col_coords = []
    for char in plaintext:
        if char in positions:
            row, col = positions[char]
            row_coords.append(row)
            col_coords.append(col)

    combined_coords = row_coords + col_coords
    ciphertext = ""
    for i in range(0, len(combined_coords), 2):
        row = combined_coords[i]
        col = combined_coords[i + 1]
        ciphertext += square[row][col]

    return ciphertext

def bifid_decrypt(ciphertext, key):
    square, positions = create_polybius_square(key)
    reverse_positions = {v: k for k, v in positions.items()}
    coords = []
    for char in ciphertext:
        for item in positions[char]:
            coords.append(item)

    row_coords = coords[:len(coords) // 2]
    col_coords = coords[len(coords) // 2:]
    plaintext = ""
    for r, c in zip(row_coords, col_coords):
        plaintext += square[r][c]

    return plaintext

plaintext = "Hello World"
key = "secretkey"
ciphertext = bifid_encrypt(plaintext, key)
decrypted_text = bifid_decrypt(ciphertext, key)
print("Ciphertext:", ciphertext)
print("Decrypted:", decrypted_text)
\end{lstlisting}

\newpage

\subsection{Trifid Cipher}

Trifid Cipher, Felix Delastelle tarafından geliştirilen çok boyutlu bir şifreleme yöntemidir. Bifid Şifreleme'nin bir uzantısı olarak, Trifid metodu metni şifrelemek için üç boyutlu bir matrisi (3x3x3) kullanır. Hem yer değiştirme hem de transpozisyon prensiplerini birleştirir, bu da daha güçlü bir şifreleme sağlar.

\subsubsection{Encryption}

\begin{enumerate}
    \item Alfabenin harfleri, 27 hücrelik bir kübe yerleştirilir.
    \item Şifrelenecek metin büyük harflere çevrilir. Harfler, küpteki pozisyonlarına göre (katman, satır, sütun) bir koordinat kümesine çevrilir.
    \item Pozisyonlar üç ayrı gruba ayrılır: Katman, Satır, Sütun koordinatları. Koordinatlar birleştirilerek sıralama değiştirilir.
    \item Yeni sıralamaya göre koordinatlar tekrar gruplanır ve küpteki harfler şifreli metni oluşturur.
\end{enumerate}

Örneğin, "ANAHTAR" anahtarı ile "ALPER" mesajını şifreleyim. 3 adet tablo:

\begin{table}[ht]
\centering
\resizebox{0.5\textwidth}{!}{%
\begin{tabular}{|l|l|l|l|}
\hline
  & 1 & 2 & 3 \\ \hline
1 & A & N & H \\ \hline
2 & T & R & B \\ \hline
3 & C & D & E \\ \hline
\end{tabular}%
}
\end{table}

\begin{table}[ht]
\centering
\resizebox{0.5\textwidth}{!}{%
\begin{tabular}{|l|l|l|l|}
\hline
  & 1 & 2 & 3 \\ \hline
1 & F & G & J \\ \hline
2 & J & K & L \\ \hline
3 & M & O & P \\ \hline
\end{tabular}%
}
\end{table}

\newpage

\begin{table}[ht]
\centering
\resizebox{0.5\textwidth}{!}{%
\begin{tabular}{|l|l|l|l|}
\hline
  & 1 & 2 & 3 \\ \hline
1 & Q & S & U \\ \hline
2 & V & W & X \\ \hline
3 & Y & Z & - \\ \hline
\end{tabular}%
}
\end{table}

Tablodaki değerlere göre "ALPER" kelimesinin tablosu:

\begin{table}[ht]
\centering
\resizebox{0.8\textwidth}{!}{%
\begin{tabular}{|l|l|l|l|l|l|}
\hline
Harf & A & L & P & E & R \\ \hline
Katman & 1 & 2 & 2 & 1 & 1 \\ \hline
Sütun & 1 & 3 & 3 & 3 & 2 \\ \hline
Satır & 1 & 2 & 3 & 3 & 2 \\ \hline
\end{tabular}%
}
\end{table}

Şimdi bu tablodaki değerleri soldan sağa doğru üçerli şekilde okuyarak Trifid tablosundaki değeriyle mesajı şifreleyelim:

\begin{itemize}
    \item 122, 1. katman 2. sütun 2. satır: R 
    \item 111, 1. katman 1. sütun 1. satır: A
    \item 333, 3. katman 3. sütun 3. satır: -
    \item 212, 2. katman 1. sütun 2. satır: J
    \item 332, 3. katman 3. sütun 2. satır: X
\end{itemize}

"ANAHTAR" anahtarı ile "ALPER" mesajının şifrelenmiş hali "RAJX" dır.

\newpage

\subsection{Vernam Cipher}

Vernam Şifreleme, 1917'de Gilbert Vernam tarafından geliştirilen ve "One-Time Pad" olarak da bilinen bir şifreleme yöntemidir. 1. Dünya Savaşında Almanların çözemeyeceği bir metod geliştirilmesi için görevlendirilen mühendis Gilbert Vernam, Joseph Mauborgne adlı bir subay ile bu yöntemi geliştirdi. Bu yöntem, bir veri ve bir anahtar arasında bir mod-2 (XOR) işlemi gerçekleştirerek veri şifreler. Düz metin (plaintext) bir bit dizisine dönüştürülür. Anahtar da aynı uzunlukta rastgele bir bit dizisi olarak oluşturulur. Şifreleme işleminde, düz metnin her biti ile anahtarın karşılık gelen biti XOR işlemi ile işlenir:

\[ C_i = P_i \oplus K_i \]

Burada, $P_i$, düz metinin i-inci biti; $K_i$, anahtarın i-inci biti; $C_i$, şifreli metnin i-inci biti.

Örneğin, "VERNAM" anahtarı ile "KARACA" kelimesini şifreleyelim. İlk olarak anahtar ve mesajdaki her bir harfin ASCII kodunun binary kodu elde edilir. Daha sonra bu anahtar ve mesajın binary kodları xor işlemine girer. ASCII tablosunda büyük-küçük harf duyarlıdır.

\begin{table}[ht]
\resizebox{\textwidth}{!}{%
\begin{tabular}{|l|l|l|l|l|l|}
\hline
Anahtar Harf & ASCII Kodu & Binary Kodu & Mesaj Harf & ASCII Kodu & Binary Kodu \\ \hline
V & 86 & 1010110 & K & 75 & 1001011 \\ \hline
E & 69 & 1000101 & A & 65 & 1000001 \\ \hline
R & 82 & 1010010 & R & 82 & 1010010 \\ \hline
N & 78 & 1001110 & A & 65 & 1000001 \\ \hline
A & 65 & 1000001 & C & 67 & 1000011 \\ \hline
M & 77 & 1001101 & A & 65 & 1000001 \\ \hline 
\end{tabular}%
}
\end{table}

\begin{table}[ht]
\resizebox{\textwidth}{!}{%
\begin{tabular}{|l|l|l|l|}
\hline
Anahtar Harf Binary & Mesaj Harf Binary & XOR İşlemi & ASCII Kodu \\ \hline
1010110 & 1001011 & 00011101 & 29 \\ \hline
1000101 & 1000001 & 00000100 & 4 \\ \hline
1010010 & 1010010 & 00000000 & 0 \\ \hline
1001110 & 1000001 & 00001111 & 15 \\ \hline
1000001 & 1000011 & 00000010 & 2 \\ \hline
1001101 & 1000001 & 00001100 & 12 \\ \hline
\end{tabular}%
}
\end{table}

"KARACA" kelimesinin şifreli hali "00011101 00000100 00000000 00001111 00000010 00001100" dir. Aynı anahtarın binary kodunu ile şifreli mesajı xor işlemine sokarak şifrelenen metin çözülür.

\subsubsection{Python Kodu}

\begin{lstlisting}[language=Python]
def vernam_encrypt(plaintext, key):
    binary_plaintext = ''.join(format(ord(char), '08b') for char in plaintext)
    binary_key = ''.join(format(ord(char), '08b') for char in key)

    if len(binary_plaintext) != len(binary_key):
        raise ValueError()

    ciphertext = ''.join('1' if p != k else '0' for p, k in zip(binary_plaintext, binary_key))
    return ciphertext

def vernam_decrypt(ciphertext, key):
    binary_key = ''.join(format(ord(char), '08b') for char in key)
    plaintext_binary = ''.join('1' if c != k else '0' for c, k in zip(ciphertext, binary_key))
    chars = [plaintext_binary[i:i+8] for i in range(0, len(plaintext_binary), 8)]
    decrypted = ''.join(chr(int(char, 2)) for char in chars)
    return decrypted

plaintext = "Hello World"
key = "secretkeyyy"
ciphertext = vernam_encrypt(plaintext, key)
decrypted_text = vernam_decrypt(ciphertext, key)
print("Ciphertext (Binary):", ciphertext)
print("Decrypted:", decrypted_text)
\end{lstlisting}

\newpage

\subsection{Hill Cipher}

Hill Cipher, 1929 yılında Lester S. Hill tarafından geliştirilen bir şifreleme yöntemidir. Bu yöntem, doğrusal cebir kullanarak matris çarpımı prensibine dayalı bir şifreleme sağlar. Hill şifreleme, blok şifreleme yöntemleri arasında yer alır ve metni belirli boyutlardaki bloklara ayırarak işlem yapar.

Şifreleme için:

\[ C = (K \times P) \text{ mod } 26 \]

Deşifreleme için:

\[ P = (K^{-1} \times C) \text{ mod } 26 \]

\subsubsection{Encryption}

\begin{enumerate}
    \item $n \times n$ boyutunda bir kare matris (anahtar) oluşturulur. Bu matris şifreleme ve deşifreleme işlemlerinde kullanılır. Anahtar matrisin determinantı 26 ile aralarında asal olmalıdır. Bu, matrisin tersinin alınabilir olmasını sağlar.
    \item Düz metin (plaintext) harfler halinde sayılara dönüştürülür. Eğer metin uzunluğu matris boyutuna tam bölünmüyorsa, boşlukları doldurmak için dolgu karakteri eklenir.
    \item Düz metin blokları, anahtar matrisi ile çarpılır ve mod 26 alınır.
\end{enumerate}

Örneğin, "java" kelimesini 2x2'lik bir anahtar ile şifreleyelim. Anahtarımız $\begin{bmatrix} 6 & 2 \\ 1 & 4 \end{bmatrix}$ olsun. İlk olarak mesaj ikili bloklara bölünür. Böylece: $java = \begin{bmatrix} j \\ a \end{bmatrix} , \begin{bmatrix} v \\ a \end{bmatrix}$ olur. 

\begin{table}[ht]
\centering
\resizebox{0.8\textwidth}{!}{%
\begin{tabular}{|l|l|l|l|l|l|l|l|l|l|l|l|l|}
\hline
A & B & C & D & E & F & G & H & I & J & K & L & M \\ \hline
0 & 1 & 2 & 3 & 4 & 5 & 6 & 7 & 8 & 9 & 10 & 11 & 12 \\ \hline
N & O & P & Q & R & S & T & U & V & W & X & Y & Z \\ \hline
13 & 14 & 15 & 16 & 17 & 18 & 19 & 20 & 21 & 22 & 23 & 24 & 25 \\ \hline
\end{tabular}%
}
\end{table}

\newpage

Sonra tabloya göre her bir harfe gelen değerlerle matris oluşturulur: $java = \begin{bmatrix} 9 \\ 0 \end{bmatrix} , \begin{bmatrix} 21 \\ 0 \end{bmatrix}$. Anahtar ile ikili bloklar çarpılır. 

\[ \begin{bmatrix} 6 & 2 \\ 1 & 4 \end{bmatrix} \cdot \begin{bmatrix} 9 \\ 0 \end{bmatrix} = \begin{bmatrix} 54 \\ 9 \end{bmatrix} \]
\[ \begin{bmatrix} 6 & 2 \\ 1 & 4 \end{bmatrix} \cdot \begin{bmatrix} 21 \\ 0 \end{bmatrix} = \begin{bmatrix} 126 \\ 21 \end{bmatrix} \]

Çıkan sonuç mod değerinden büyük olduğu için 26'ya göre modu alınır.

\[ \begin{bmatrix} 54 \\ 9 \end{bmatrix} \text{ mod } 26 = \begin{bmatrix} 2 \\ 9 \end{bmatrix} \]
\[ \begin{bmatrix} 126 \\ 21 \end{bmatrix} \text{ mod } 26 = \begin{bmatrix} 22 \\ 9 \end{bmatrix} \]

Bulunan değerlerin tablodaki harf karşılığı bize şifreli mesajı verir.

\[ \begin{bmatrix} 2 \\ 9 \end{bmatrix} = \begin{bmatrix} c \\ j \end{bmatrix} \]
\[ \begin{bmatrix} 22 \\ 21 \end{bmatrix} = \begin{bmatrix} w \\ j \end{bmatrix} \]

Böylece "java" mesajının şifreli hali "cjwj" olur. Şimdi ise bu şifreli mesajı çözelim. Öncelikle anahtar matrisinin tersi alınır; anahtar matrisinin tersi ile kendisinin çarpımı bize birim matrisi verir. Anahtar matrisin tersinde negatif değer varsa mod yani 26 değeri eklenir. Anahtar matrisin tersini hesaplarsak:

\[ A^{-1} = (ad - bc)^{-1} \begin{bmatrix} d & -b \\ -c & a \end{bmatrix} \]
\[ A^{-1} = (6 \cdot 4 - 2 \cdot 1)^{-1} \begin{bmatrix} 4 & -2 \\ -1 & 6 \end{bmatrix} \text{ mod } 26 \]
\[ A^{-1} = \frac{1}{22} \cdot \begin{bmatrix} 4 & -2 \\ -1 & 6 \end{bmatrix} \text{ mod } 26 \]

Matrisin tersi bulunduktan sonra matris ile şifreli mesaj çarpılır. Bulunan değerlerin 26'ya göre modu alınır. Elde edilen değerlerin tablodaki harf karşılıkları alındığında metin çözülmüş olur.

\subsubsection{Python Kodu}

\begin{lstlisting}[language=Python]
import numpy as np

def mod_inverse(a, m):
    a = a % m
    for x in range(1, m):
        if (a * x) % m == 1:
            return x
    return None

def hill_encrypt(plaintext, key_matrix):
    n = key_matrix.shape[0]
    plaintext = plaintext.upper().replace(" ", "")
    if len(plaintext) % n != 0:
        plaintext += 'X' * (n - len(plaintext) % n)
    
    plaintext_numbers = [ord(char) - ord('A') for char in plaintext]
    plaintext_blocks = np.array(plaintext_numbers).reshape(-1, n)
    
    ciphertext = []
    for block in plaintext_blocks:
        encrypted_block = np.dot(key_matrix, block) % 26
        ciphertext.extend(encrypted_block)

    encrypted = ''.join(chr(num + ord('A')) for num in ciphertext)
    return encrypted

def hill_decrypt(ciphertext, key_matrix):
    n = key_matrix.shape[0]
    ciphertext_numbers = [ord(char) - ord('A') for char in ciphertext]
    ciphertext_blocks = np.array(ciphertext_numbers).reshape(-1, n)
    
    det = int(round(np.linalg.det(key_matrix)))
    det_inv = mod_inverse(det, 26)
    key_matrix_inv = np.linalg.inv(key_matrix) * det
    key_matrix_inv = (key_matrix_inv * det_inv) % 26
    key_matrix_inv = np.round(key_matrix_inv).astype(int) % 26
    
    plaintext = []
    for block in ciphertext_blocks:
        decrypted_block = np.dot(key_matrix_inv, block) % 26
        plaintext.extend(decrypted_block)

    decrypted = ''.join(chr(num + ord('A')) for num in plaintext)
    return decrypted

key_matrix = np.array([[6, 24, 1], [13, 16, 10], [20, 17, 15]])
plaintext = "Hello World"
ciphertext = hill_encrypt(plaintext, key_matrix)
decrypted_text = hill_decrypt(ciphertext, key_matrix)
print("Ciphertext:", ciphertext)
print("Decrypted:", decrypted_text)
\end{lstlisting}

\newpage

\subsection{Bible Code}

Bible Code (İncil Kodu), klasik bir şifreleme yöntemi olmaktan çok, belirli bir metin içerisinden gizli mesajlar ya da kodlar bulma amacıyla kullanılan bir yöntemdir. Tarihte, İncil gibi büyük metinlerde gizli mesajların bulunabileceği inancı üzerine kurgulanmıştır. Bu yöntem, metin içerisindeki harflerin belirli bir desenle seçilmesi ve bir mesaj oluşturulması üzerine dayanır.

\subsubsection{Encryption}

\begin{enumerate}
    \item Şifreleme ve çözme işlemleri için bir kaynak metin seçilir. Bu genelde İncil gibi uzun bir metindir.
    \item Şifrelenecek veriye göre bir desen (örneğin her 5. harfi seçmek gibi) belirlenir.
    \item Şifrelenecek mesajın her bir harfi, kaynak metindeki bir pozisyonla eşleştirilir.
    \item Kaynak metin üzerinden harflerin sıralı bir şekilde bulunması sağlanır.
\end{enumerate}

\subsubsection{Python Kodu}

\begin{lstlisting}[language=Python]
def bible_code_encrypt(message, text, step):
    message = message.upper().replace(" ", "")
    text = text.upper().replace(" ", "").replace("\n", "")
    
    indices = []
    current_index = 0
    
    for char in message:
        while current_index < len(text):
            if text[current_index] == char:
                indices.append(current_index)
                current_index += step
                break
            current_index += 1
    
    return indices

def bible_code_decrypt(indices, text):
    text = text.upper().replace(" ", "").replace("\n", "")
    message = ''.join([text[i] for i in indices])
    return message

source_text = ""
message_to_encrypt = "God"
step_size = 5

encrypted_indices = bible_code_encrypt(message_to_encrypt, source_text, step_size)
decrypted_message = bible_code_decrypt(encrypted_indices, source_text)
print("Encrypted (Indexes):", encrypted_indices)
print("Decrypted:", decrypted_message)
\end{lstlisting}

\newpage

\subsection{Base64}

Base64, ikili verileri (binary data) ASCII formatına dönüştürmek için kullanılan bir kodlama yöntemidir. Şifreleme değil, bir kodlama yöntemidir ve esas amacı, veriyi taşınabilir ve okunabilir hale getirmektir. E-posta sistemlerinde, URL'lerde veya JSON formatında veriyi taşırken kullanılır. Veriyi 6 bitlik gruplara böler ve bu grupları bir tabloya göre ASCII karakterlerine dönüştürür. Alfabetik harfler (A-Z, a-z), rakamlar (0-9), +, / karakterlerini kullanır. 64 farklı karakter kullandığı için "Base64" adını almıştır. Eksik bitleri tamamlamak için "=" karakteri ile dolgu yapılır.

\subsubsection{Encryption}

\begin{enumerate}
    \item Kodlanacak veri önce ASCII değerlerine, ardından ikili (binary) formatına dönüştürülür.
    \item Oluşan binary veri 6 bitlik gruplara bölünür. Eğer toplam uzunluk 6'nın katı değilse, veri 0 eklenerek tamamlanır. 
    \item 6 bitlik gruplar, Base64 tablosundaki karakterlere dönüştürülür.
    \item Eğer veri tam 3 byte (24 bit) değilse, eksik kısımlar eşitlik (=) karakteriyle doldurulur.
\end{enumerate}

Base64 tablosunda, 0-25 arası indekslerde A-Z, 26-51 arası indekslerde a-z, 52-61 arası indekslerde 0-9, 62. indekste "+" ve 63. indiste "/" bulunur. Şifrelenecek mesaj önce üçerli gruplara bölünür. Bunun nedeni her karakterin 8 bit olması ve bu blokların 6 bitlik yeni bloklara bölünecek olmasıdır. 6 ve 8'ın EKOK'u 24'tür. Her üçerli blokta 24 bit bulunur. 24 bit ile 6 bitlik 4 blok oluşturulur. Örneğin, Base64 ile "KARACA" kelimesini şifreleyelim. "KARACA" mesajını "KAR" ve "ACA" olmak üzere ikiye ayıralım. Her parçada 3 karakter var, her karakter 8 bit ise 24 bit elde ettik. Bu 24 biti de 6 bitlik 4 gruba böleceğiz.

\begin{table}[ht]
\centering
\resizebox{0.5\textwidth}{!}{%
\begin{tabular}{|l|l|l|}
\hline
  & ASCII Değeri & Binary Değeri \\ \hline
K & 75 & 01001011 \\ \hline
A & 65 & 01000001 \\ \hline
R & 82 & 01010010 \\ \hline
A & 65 & 01000001 \\ \hline
C & 67 & 01000011 \\ \hline
A & 65 & 01000001 \\ \hline
\end{tabular}%
}
\end{table}

Böylece:

\begin{itemize}
    \item KAR: 010010110100000101010010
    \item ACA: 010000010100001101000001
\end{itemize}

Elde edildi. Şimdi bu mesajları 6 bitlik gruplara bölelim

\begin{itemize}
    \item KAR: 010010 110100 000101 010010
    \item ACA: 010000 010100 001101 000001
\end{itemize}

Bu değerlerin ASCII değerinin tablodaki karşılığını alarak mesajı şifreleyelim.

\begin{table}[ht]
\centering
\resizebox{0.5\textwidth}{!}{%
\begin{tabular}{|l|l|l|}
\hline
Binary Değeri & Decimal Değeri & Tablodaki Karakter \\ \hline
010010 & 18 & S \\ \hline
110100 & 52 & 0 \\ \hline
000101 & 5  & F \\ \hline
010010 & 18 & S \\ \hline
010000 & 16 & Q \\ \hline
010100 & 20 & U \\ \hline
001101 & 13 & N \\ \hline
000001 & 1  & B \\ \hline
\end{tabular}%
}
\end{table}

"KARACA" kelimesinin base64 ile şifrelenmiş hali "S0FSQUNB" elde edildi. Şifre çözme aşamasında ise bu karakterlerin binary karşılıklarının 8 bitlik gruplar halinde bölünür ve bu grupların decimal değerinin tablodaki karşılığı elde edilerek mesaj çözülür.

\subsubsection{Python Kodu}

\begin{lstlisting}[language=Python]
import base64

def base64_encode(data):
    byte_data = data.encode('utf-8')
    encoded_data = base64.b64encode(byte_data)
    return encoded_data.decode('utf-8')

def base64_decode(encoded_data):
    byte_data = encoded_data.encode('utf-8')
    decoded_data = base64.b64decode(byte_data)
    return decoded_data.decode('utf-8')

plaintext = "Hello World"
encoded = base64_encode(plaintext)
decoded = base64_decode(encoded)
print("Encoded:", encoded)
print("Decoded:", decoded)
\end{lstlisting}

\newpage

\subsection{ROT13 Cipher}

ROT13 (Rotation by 13 places) basit bir şifreleme yöntemidir. Her harfi, alfabedeki 13. harfe kaydırarak şifreler. Eğer bir harf A ile M arasında ise, harf 13 kaydırılır; eğer N ile Z arasında ise yine 13 kaydırılır. Bu yöntem, şifreyi hem şifreler hem de çözer, çünkü alfabede 26 harf olduğundan, 13 kaydırma işleminden sonra aynı harfe geri dönülür. Bu özellik, ROT13'ün kolayca çözülmesini sağlar.

\subsubsection{Python Kodu}

\begin{lstlisting}[language=Python]
import string

def rot13_encrypt_decrypt(text):
    rot13_table = str.maketrans(
        string.ascii_lowercase + string.ascii_uppercase,
        string.ascii_lowercase[13:] + string.ascii_lowercase[:13] +
        string.ascii_uppercase[13:] + string.ascii_uppercase[:13]
    )
    return text.translate(rot13_table)

plaintext = "Hello World"
encrypted_text = rot13_encrypt_decrypt(plaintext)
decrypted_text = rot13_encrypt_decrypt(encrypted_text)
print("Encrpyted:", encrypted_text)
print("Decrypted:", decrypted_text)
\end{lstlisting}

\begin{table}[ht]
\centering
\resizebox{0.8\textwidth}{!}{%
\begin{tabular}{|l|l|l|l|l|l|l|l|l|l|l|l|l|}
\hline
A & B & C & D & E & F & G & H & I & J & K & L & M \\ \hline
0 & 1 & 2 & 3 & 4 & 5 & 6 & 7 & 8 & 9 & 10 & 11 & 12 \\ \hline
N & O & P & Q & R & S & T & U & V & W & X & Y & Z \\ \hline
13 & 14 & 15 & 16 & 17 & 18 & 19 & 20 & 21 & 22 & 23 & 24 & 25 \\ \hline
\end{tabular}%
}
\end{table}

Örneğin, "ALPER" mesajının ROT13 ile şifrelenmiş hali tabloya göre "NYCRE" dir.

\newpage

\subsection{Lehmer Code}

Lehmer Code, sıralı bir küme içerisindeki öğelerin sırasını temsil etmek için kullanılan bir kodlama sistemidir. Sıralama ve permütasyonlarla ilişkili işlemlerde kullanılır. Lehmer kodu, bir permütasyonun sırasını ifade etmek için her öğe için "daha küçük" öğelerin sayısını belirler. Bu yöntem, belirli bir öğenin sırasını hesaplamak için kullanılır.

\subsubsection{Python Kodu}

\begin{lstlisting}[language=Python]
def lehmer_code_encryption(perm):
    n = len(perm)
    lehmer = []
    for i in range(n):
        count = 0
        for j in range(i + 1, n):
            if perm[j] < perm[i]:
                count += 1
        lehmer.append(count)
    return lehmer

def lehmer_code_decryption(lehmer):
    n = len(lehmer)
    perm = []
    elements = list(range(1, n + 1))
    
    for i in range(n):
        index = lehmer[i]
        perm.append(elements.pop(index))
    return perm

perm = [3, 1, 2]
encrypted = lehmer_code_encryption(perm)
decrypted = lehmer_code_decryption(encrypted)
print("Encrypted:", encrypted)
print("Decrypted:", decrypted)
\end{lstlisting}

\newpage

\subsection{Linear Cipher}

Linear Cipher, matematiksel bir doğrusal eşitlik kullanarak veriyi şifreler ve şifresini çözer. Şifreleme işlemi, bir matris çarpımı ve bir kaydırma gibi doğrusal işlemleri içerir. Şifreleme işlemi tersine çevrilebilir olmalıdır. Bu, matrisin tersinin alınabilmesi gerektiği anlamına gelir. Şifreleme için:

\[ y = Ax + b \]

Burada, $y$ şifreli metni, $A$ şifrelemek için kullanılacak matris ve $b$ kaydırma vektörünü temsil eder. Şifreyi çözmek için:

\[ x = A^{-1} (y - b) \]

Şifreyi çözmek için şifreleme kullanılan matrisin tersi hesaplanır.

\subsubsection{Encrpytion}

Örneğin, "ALPER" kelimesini şifreleyelim. $A = 5$ ve $b = 9$ olsun. 

\begin{table}[ht]
\centering
\resizebox{0.8\textwidth}{!}{%
\begin{tabular}{|l|l|l|l|l|l|l|l|l|l|l|l|l|}
\hline
A & B & C & D & E & F & G & H & I & J & K & L & M \\ \hline
0 & 1 & 2 & 3 & 4 & 5 & 6 & 7 & 8 & 9 & 10 & 11 & 12 \\ \hline
N & O & P & Q & R & S & T & U & V & W & X & Y & Z \\ \hline
13 & 14 & 15 & 16 & 17 & 18 & 19 & 20 & 21 & 22 & 23 & 24 & 25 \\ \hline
\end{tabular}%
}
\end{table}

Tabloya göre "ALPER" kelimesinin her bir harfi için:

\begin{itemize}
    \item A harfinin tablodaki değeri 0, $y = (5 \cdot 0 + 9) \text{ mod } 26 = 9$, yani "J" harfine karşılıktır.
    \item L harfinin tablodaki değeri 11, $y = (5 \cdot 11 + 9) \text{ mod } 26 = 12$, yani "M" harfine karşılıktır.
    \item P harfinin tablodaki değeri 15, $y = (5 \cdot 15 + 9) \text{ mod } 26 = 6$, yani "G" harfine karşılıktır.
    \item E harfinin tablodaki değeri 4, $y = (5 \cdot 4 + 9) \text{ mod } 26 = 3$, yani "D" harfine karşılıktır.
    \item R harfinin tablodaki değeri 17, $y = (5 \cdot 17 + 9) \text{ mod } 26 = 16$, yani "Q" harfine karşılıktır.
\end{itemize}

"ALPER" kelimesinin şifrelenmiş hali "JMGDQ" dir. Şifreyi çözmek için ilk önce $A$'nın modüler tersi bulunmalıdır. 5'in 26'ya göre modüler tersi 21'dir, böylece $A^{-1} = 21$ olur. Eğer işlem negatif çıkarsa 26 ya göre modu eklenerek pozitif yapılır. Formülü uygularsak:

\begin{itemize}
    \item J harfinin tablodaki değeri 9, $y = (21 \cdot (9 - 9)) \text{ mod } 26 = 0$, yani "A" harfine karşılıktır.
    \item M harfinin tablodaki değeri 12, $y = (21 \cdot (12 - 9)) \text{ mod } 26 = 11$, yani "L" harfine karşılıktır.
    \item G harfinin tablodaki değeri 6, $y = (21 \cdot (6 - 9)) \text{ mod } 26 = 15$, yani "P" harfine karşılıktır.
    \item D harfinin tablodaki değeri 3, $y = (21 \cdot (3 - 9)) \text{ mod } 26 = 4$, yani "E" harfine karşılıktır.
    \item Q harfinin tablodaki değeri 16, $y = (21 \cdot (16 - 9)) \text{ mod } 26 = 17$, yani "R" harfine karşılıktır.
\end{itemize}

Şifreyi çözerek "JMGDQ" mesajından tekrar "ALPER" mesajını elde ettik.

\subsubsection{Python Kodu}

\begin{lstlisting}[language=Python]
import numpy as np

def linear_encrypt(plaintext, A, b):
    numerical_text = [ord(char) - ord('A') for char in plaintext.upper()
                      if char.isalpha()]
    x = np.array(numerical_text)
    y = np.dot(A, x) + b
    return y

def linear_decrypt(ciphertext, A, b):
    A_inv = np.linalg.inv(A).astype(int)
    x = np.dot(A_inv, ciphertext - b).astype(int)
    plaintext = "".join(chr(num + ord("A")) for num in x)
    return plaintext

plaintext = "AK"
A = np.array([[3, 2], [2, 8]])
b = np.array([1, 4])
ciphertext = linear_encrypt(plaintext, A, b)
decrypted = linear_decrypt(ciphertext, A, b)
print("Encrypted:", ciphertext)
print("Decrypted:", decrypted)
\end{lstlisting}

\newpage

\subsection{Rail Fence Technique}

Rail Fence, bir transpozisyon şifreleme tekniğidir. Veriyi belirli bir sayıda ray boyunca zikzak şeklinde yazmayı ve ardından bu raylarda yer alan karakterleri birleştirerek şifrelenmiş metni oluşturmayı içerir.

\subsubsection{Encrpytion}

\begin{enumerate}
    \item Şifrelenecek metin bir dizi ray boyunca zikzak şeklinde yazılır.
    \item Raylardaki karakterler sırayla birleştirilerek şifrelenmiş metin elde edilir.
\end{enumerate}

Örneğin "ALPERKARACA" kelimesini şifreleyelim. Ray sayımız 2 olsun.

\begin{table}[ht]
\centering
\resizebox{0.8\textwidth}{!}{%
\begin{tabular}{|l|l|l|l|l|l|l|l|l|l|l|}
\hline
A &   & P &   & R &   & A &   & A &   & A \\ \hline
  & L &   & E &   & K &   & R &   & C &   \\ \hline
\end{tabular}%
}
\end{table}

Şifreli metin, "APRAAALEKRC" elde edilmiştir.

\subsubsection{Python Kodu}

\begin{lstlisting}[language=Python]
def rail_fence_encrypt(plaintext, num_rails):
    rails = ['' for _ in range(num_rails)]
    rail = 0
    direction = 1  # 1: Down, -1: Up

    for char in plaintext:
        rails[rail] += char
        rail += direction
        if rail == 0 or rail == num_rails - 1:
            direction *= -1

    return ''.join(rails)

def rail_fence_decrypt(ciphertext, num_rails):
    rail_lengths = [0] * num_rails
    rail = 0
    direction = 1

    for _ in ciphertext:
        rail_lengths[rail] += 1
        rail += direction
        if rail == 0 or rail == num_rails - 1:
            direction *= -1

    rails = []
    idx = 0
    for length in rail_lengths:
        rails.append(list(ciphertext[idx:idx + length]))
        idx += length

    result = []
    rail = 0
    direction = 1

    for _ in ciphertext:
        result.append(rails[rail].pop(0))
        rail += direction
        if rail == 0 or rail == num_rails - 1:
            direction *= -1

    return ''.join(result)

num_rails = 2
plaintext = "ALPERKARACA"
ciphertext = rail_fence_encrypt(plaintext, num_rails)
decrypted = rail_fence_decrypt(ciphertext, num_rails)
print("Encrypted:", ciphertext)
print("Decrypted:", decrypted)
\end{lstlisting}

\newpage

\subsection{Row-Column Transposition Cipher}

Row-Column Transposition Cipher, metin şifrelemek için kullanılan bir transpozisyon şifreleme yöntemidir. Bu yöntemde metin, belirli bir satır ve sütun düzenine göre bir tabloya yerleştirilir. Şifreleme işlemi, tablodaki karakterleri sırasıyla sütunlar veya satırlar üzerinden okumayı içerir.

\subsubsection{Encryption}

\begin{enumerate}
    \item Şifrelenecek metin uzunluğu, belirlenen tablo boyutlarına uygun olacak şekilde düzenlenir. Gerekirse metne dolgu karakterleri eklenir.
    \item Metin, tabloya satır satır yerleştirilir.
    \item Tablodaki metin sütun sütun okunarak şifrelenmiş metin oluşturulur.
\end{enumerate}

Örneğin, "ALPERKARACA" mesajını şifreleyelim. 4 sütun ve 3 satır olsun.

\begin{table}[ht]
\centering
\resizebox{0.8\textwidth}{!}{%
\begin{tabular}{|l|l|l|l|}
\hline
A & L & P & E \\ \hline
R & K & A & R \\ \hline
A & C & A & X \\ \hline
\end{tabular}%
}
\end{table}

Şifrelenmiş metin, "ARALKCPAAERX" elde edilmiştir.

\subsubsection{Python Kodu}

\begin{lstlisting}[language=Python]
import math

def row_column_encrypt(plaintext, num_cols):
    num_rows = math.ceil(len(plaintext) / num_cols)
    padded_length = num_rows * num_cols
    plaintext = plaintext.ljust(padded_length, 'X')

    ciphertext = ""
    for col in range(num_cols):
        for row in range(num_rows):
            idx = row * num_cols + col
            ciphertext += plaintext[idx]

    return ciphertext

def row_column_decrypt(ciphertext, num_cols):
    num_rows = math.ceil(len(ciphertext) / num_cols)
    plaintext = [""] * len(ciphertext)
    idx = 0
    for col in range(num_cols):
        for row in range(num_rows):
            if idx < len(ciphertext):
                plaintext[row * num_cols + col] = ciphertext[idx]
                idx += 1

    return "".join(plaintext).rstrip('X')

num_cols = 4
plaintext = "ALPERKARACA"
ciphertext = row_column_encrypt(plaintext, num_cols)
decrypted = row_column_decrypt(ciphertext, num_cols)
print("Encrypted:", ciphertext)
print("Decrypted:", decrypted)
\end{lstlisting}

\newpage