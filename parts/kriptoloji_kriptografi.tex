\section{Kriptoloji, Kriptografi, Kriptoanaliz}

\subsection{Kriptoloji}

Kriptoloji, bilgi güvenliği ile ilgilenen bilim dalıdır. Kriptoloji, sadece bilgilerin gizlenmesini değil, aynı zamanda bu bilgilerin bütünşüğünü, kimlik doğrulamasını ve reddedilemezliğini de ele alır. Kriptoloji terimi, Yunanca "kryptos" (gizli) ve "logos" (bilim) kelimelerinden türetilmiştir. Kriptoloji, iki temel alt disiplini kapsar:

\begin{itemize}
    \item \textbf{Kriptografi}: Verilerin gizliliğini sağlamak ve korumak için kullanılan yöntemleri inceleyen bilim dalıdır.
    \item \textbf{Kriptoanaliz}: Şifrelenmiş mesajların güvenliğini analiz eden ve bu mesajları kırmaya veya çözmeye çalışan bilim dalıdır.
\end{itemize}

\subsection{Kriptografi}

Kriptografi, verilerin şifrelenmesi ve şifre çözme yöntemlerini geliştirir. Bilgiyi yetkisiz erişime karşı korumak için matematiksel algoritmalar ve teknikler kullanır. Amacı, bir mesajı sadece belirli kişilerin anlayabileceği şekilde dönüştürmek (şifrelemek) ve daha sonra bu mesajın orijinal haline geri getirilmesini (şifre çözme) sağlamaktır.

\begin{itemize}
    \item \textbf{Gizlilik (Confidentiality)}: Bilgilerin sadece yetkili kişilerce okunabilmesini sağlamak.
    \item \textbf{Bütünlük (Integrity)}: Bilgilerin iletim sırasında değiştirilmediğinden emin olmak.
    \item \textbf{Kimlik Doğrulama (Authentication)}: Mesajın kimden geldiğini doğrulamak.
    \item \textbf{Reddedilemezlik (Non-repudiation)}: Bir işlemi veya mesajı gönderen kişinin, bu işlemi gerçekleştirdiğini inkar edememesini sağlamak.
\end{itemize}

\subsection{Kriptoanaliz}

Kriptoanaliz, kriptografik sistemlerin güvenliğini test etme sürecidir. Kriptoanalistler, şifrelenmiş bilgileri çözmek veya zayıf yönleri bulmak için çalışırlar. Bu alan, güvenlik sistemlerinin ne kadar dayanıklı olduğunu test eder ve olası saldırılara karşı dayanıklılıklarını değerlendirir. Başarılı bir kriptoanaliz, şifrelenmiş bir mesajı şifreleme anahtarını bilmeden çözebilir.

\newpage