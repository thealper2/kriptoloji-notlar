\section{Supply Chain}

Supply Chain (Tedarik Zinciri), ürünlerin veya hizmetlerin hammaddeden tüketiciye ulaşana kadar geçtiği tüm aşamaları kapsayan bir sistemdir. Blockchain tabanlı Supply Chain, bu sürecin daha şeffaf, güvenilir ve verimli bir şekilde yönetilmesini sağlar. Blockchain teknolojisi, dağıtık bir defter sistemi kullanarak tedarik zincirindeki tüm işlemleri değişmez, güvenli ve izlenebilir bir şekilde kaydeder.

\begin{itemize}
    \item Tedarik zincirindeki her bir adımı kayıt altına alır ve bu kayıtlara tüm yetkili tarafların erişimini sağlar. Bu, sahteciliği önler ve güven artırır.
    \item Ürünlerin nereden geldiğini, hangi aşamalardan geçtiğini ve tüketiciye nasıl ulaştığını izlemeyi mümkün kılar.
    \item İşlemlerin ve belgelerin dijital olarak doğrulanması sayesinde insan hatası ve sahtecilik riski azalır.
    \item Geleneksel tedarik zincirinde kullanılan karmaşık belge süreçlerini dijitalleştirir ve hızlandırır.
    \item Blockchain’in değiştirilemez kayıt özelliği sayesinde, tedarik zincirine dahil olan tüm taraflar birbirlerine güvenmek yerine sisteme güvenir.
\end{itemize}

\newpage