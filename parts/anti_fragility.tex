\section{Anti-Fragility}

Anti-Fragility, bir sistemin olumsuz koşullar altındayken zarar görmemesi aksine bu durumdan güçlenerek çıkması anlamına gelir. Kavram, Nicholas Nassim Taleb'in "Antifragile: Things That Gain From Disorder" kitabında tanımlanmıştır. Blockchain teknolojisinde ise, ağın saldırılar, hatalar veya değişiklikler gibi olumsuz durumlarla karşılaştığında daha dirençli ve güçlü hale gelmesi anlamına gelir.

\begin{itemize}
    \item Blockchain ağında tüm veriler birden fazla düğüm üzerinde saklanır. Birkaç düğüm başarısız olsa bile sistemin genel işlevselliği etkilenmez. Ağa yapılan saldırılar, zayıf noktaları ortaya çıkarır ve bu noktaların güçlendirilmesini sağlar.
    \item Konsensüs algoritmaları, ağın kötü niyetli aktörlere karşı kendini korumasını sağlar.
    \item Blockchain ağı, geçmiş saldırıları analiz ederek güvenlik açıklarını kapatır.
    \item Hard fork ve soft fork mekanizmaları, ağın kendini yenilemesine ve iyileştirmesine olanak tanır.
    \item Blockchain projeleri genellikle açık kaynaklıdır ve dünya çapında bir geliştirici topluluğu tarafından desteklenir. Bu topluluk, olası tehditlere karşı hızlı bir şekilde çözüm üretebilir ve sistemi daha güçlü hale getirebilir.
    \item Blockchain'de kullanılan kriptografik algoritmalar, ağın dış tehditlere karşı dayanıklılığını artırır. Eğer bir zayıflık bulunursa, daha güçlü algoritmalar veya protokollerle değiştirilebilir.
\end{itemize}

\newpage