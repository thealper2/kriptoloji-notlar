\section{Gizli Anahtarlı Şifreleme (Symmetric Encryption)}

Simetrik şifrelemede, aynı anahtar hem şifreleme hem de şifre çözme işlemleri için kullanılır. Veri, bir şifreleme algoritması ve bir gizli anahtar kullanılarak şifrelenir. Şifrelenmiş veri, aynı gizli anahtarı bilen bir alıcıya gönderilir. Alıcı, aynı anahtarı kullanarak veriyi çözer ve orijinal haline ulaşır. Çok hızlıdır. Anahtar paylaşımı güvenli bir şekilde yapılmazsa sistem kırılabilir Her iki taraf da aynı gizli anahtara sahip olmalıdır. VPN bağlantılarında kullanılır. 

\begin{itemize}
    \item \textbf{Blok Şifreleme}: Veriyi sabit uzunlukta bloklara bölerek şifreler. Örneğin; DES, 3DES, AES.
    \item \textbf{Akış Şifreleme}: Veriyi tek tek bitler veya baytlar halinde şifreler. Örneğin; RC4, Salsa20, ChaCha20.
    \item \textbf{Hybrid Yaklaşımlar}: Akış ve blok şifrelemenin avantajlarını birleştirir. Çoğu modern protokolde hibrit modeller tercih edilir.
\end{itemize}

Stream Cipher, veriyi bit bazında işler. Yani, her bir veri bitini (1 veya 0) tek tek şifreler. Veriyi sürekli (stream) olarak işler ve her bit için bir anahtar akışı oluşturur. Şifreleme ve şifre çözme işlemleri aynıdır. Şifreleme işlemi nasıl yapılırsa şifre çözme işlemi de aynı şekilde yapılır. Veri sürekli akış halinde olduğu için verilerin sırası önemli değildir. Anahtar akışı anahtarın bir parçası veya şifreleme algoritması tarafından üretilen bir sıralı sayı ile oluşuturulur. Aynı anahtar birden fazla bitin şifrelenmesinde kullanılabilir. Bit bitin hatalı olması sadece o bitin deşifre edilmesini etkiler. Hatalı bir bit şifreli veride yalnızca bir bitin yanlış çözülmesine yol açar. Hızlıdır. Veri uzunluğu belirsiz veya çok büyük olduğunda verimlidir. Ses ve video akışları, Wi-Fi, GSM şifrelemeleri gibi alanlarda kullanılır. Düşük donanım gereksinimlerine sahip cihazlarda tercih edilir.

Block Cipher, veriyi sabit uzunluktaki bloklar halinde işler. Veri birden fazla bitin bir bloğu olarak şifrelenir ve her blok, şifreleme işlemi sırsaında aynı anahtarı kullanır. Yavaştır. Şifreleme ve şifre çözme işlemleri ters sıralarda yapılır. Şifreleme ve şifre çözme algoritmaları birbirinden farklıdır. Sabit uzunluktaki veri bloklarını şifrelerken anahtar her blok için aynı kalır. Eğer bir blokta hata oluşursa, tüm blok etkilenir. Bu, hataların daha geniş bir alana yayılmasına yol açar. Dosya şifrelemeleri gibi alanlarda kullanılır. Daha güvenlidir.

\subsection{Feistel Cipher}

Feistel Yapısı, modern blok şifreleme algoritmalarının temelinde yatan bir yapıdır. Bu yapı, veriyi birden fazla turda işleyerek şifreleme veya deşifreleme işlemi gerçekleştirir. Feistel Yapısı'nın en önemli özelliği, aynı algoritmanın hem şifreleme hem de deşifreleme için kullanılabilmesidir. 

\subsubsection{Encrpytion}

\[ L_{i + 1} = R_i \]
\[ R_{i + 1} = L_i \oplus F(R_i, K_i) \]

Burada, $F$ karmaşıklık ekleyen bir fonksiyondur ve $K_i$, tur anahtarıdır.

\begin{enumerate}
    \item Açık metin eşit boyutta iki parçaya bölünür: Sol ($L$) ve Sağ ($R$).
    \item Birden fazla tur uygulanır. Her turda, bir tur anahtarı kullanılır ve bu anahtar, her tur için farklı olabilir. Sol ve sağ parçalar birbiriyle değiştirilir ve bir fonksiyon (F) uygulanır. Bu fonksiyon, karmaşıklık katmak için tasarlanmıştır. Giriş verisi üzerinde matematiksel veya bit manipülasyonu işlemleri yapar.
    \item Tüm turlar tamamlandıktan sonra $R$ ve $L$ birleştirilir.
    \item Şifre çözmek için, şifreleme sırasında uygulanan işlemler ters sırada gerçekleştirilerek yapılır.
\end{enumerate}

Örneğin, "00111010" anahtarını şifreleyelim. $K_0 = 0111$ ve $K_1 = 1001$ olsun. Fonksiyon olarak OR işlemini kullanalım. Tur sayımızı (round) 2 olsun.

\begin{enumerate}
    \item Birinci turda;
    \begin{itemize}
        \item Metin iki parçaya bölünür: "0011" ve "1010".
        \item $L_0 = 0011$ ve $R_0 = 1010$ için $R_1 = L_0 \oplus F(R_0, K_0)$ işlemi yapılır. $F(R_0, K_0) = 1010 OR 0111 = 1111$ elde edilir. Şimdi bu değer ile $L_0$ XOR işlemine girer, $L_0 \oplus 1111 = 1010 \oplus 1111 = 1100$. Bu değer bizim yeni $R_1$ değerimizdir. Artık $L_1 = 1010$ ve $R_1 = 1100$ oldu. 
    \end{itemize}
    \item İkinci turda;
    \begin{itemize}
        \item $R_2 = L_1 \oplus F(R_1, K_1)$ için $F(R_1, K_1) = 1100 OR 1001 = 1101$ elde edilir. Bu değer ile $L_1$ XOR işlemine girer, $L_1 = \oplus 1101 = 1010 \oplus 1101 = 0111$. Bu değer bizim yeni $R_2$ değerimizdir. Artık $L_2 = 1100$ ve $R_2 = 0111$ oldu. 
    \end{itemize}
    \item Swap işlemi yapılır, yani $L_2 = 0111$ ve $R_2 = 1100$ olur. Böylece şifrelenmiş veri "01111100" elde edilir.
    \item Şifre çözme işlemi için de aynı işlemler yapılır. Fakat bu sefer anahtarlar tersten kullanılır yani ilk olarak $K_1$ anahtarı kullanılır.
\end{enumerate}

\subsubsection{Python Kodu}

\begin{lstlisting}[language=Python]
def feistel_encrypt(plaintext, keys):
    L = plaintext >> 32
    R = plaintext & 0xFFFFFFFF
    for key in keys:
        F_output = R ^ key
        new_L = R
        new_R = L ^ F_output
        L = new_L
        R = new_R

    ciphertext = (R << 32) | L
    return ciphertext

def feistel_decrypt(ciphertext, keys):
    L = ciphertext >> 32
    R = ciphertext & 0xFFFFFFFF
    for key in keys[::-1]:
        F_output = R ^ key
        new_L = R
        new_R = L ^ F_output
        L = new_L
        R = new_R

    decrypted = (R << 32) | L
    return decrypted

plaintext = 0x0000003A
keys = [0x00000007, 0x00000009]
ciphertext = feistel_encrypt(plaintext, keys)
decrypted = feistel_decrypt(ciphertext, keys)
print("Encrypted:", ciphertext)
print("Decrypted:", decrypted)
\end{lstlisting}

\newpage

\subsection{DES (Data Encryption Standard)}

DES, 1970'lerde IBM tarafından geliştirilmiş ve ABD hükümeti tarafından standart olarak kabul edilmiş bir blok şifreleme algoritmasıdır. Veriyi 64 bitlik bloklar halinde işler ve 56 bit bir anahtar kullanır. Her bloğun son bit parity için kullanıldığından şifreleme işleminde kullanılmaz, bu yüzden 56 bit anahtar kullanır. 56-bit anahtar uzunluğu kısa olduğundan brute-force saldırılarına karşı zayıftır. Şifreleme ve şifre çözme için Feistel yapısını kullanır ve şifreleme işlemi toplamda 16 tur gerçekleştirir. Her turda farklı bir alt anahtar kullanarak veriyi işler. 

Açık anahtarlı şifrelemenin kurucularından Whitfield Diffie ve Martin Hellman, DES'in ticari amaçlar için güvenli olduğunu fakat istihbat için kullanılmaması gerektiğini, algoritmanın saldırılara karşı dayanıksız olduğunu ifade ettiler. Anahtar boyutunun kısa olması ve algoritma içerisindeki S kutularının güvenirliliğinin az olmasından söz ettiler. Buna rağmen DES algoritması kullanılmaya devam edildi. 1977 yılında Whitfield Diffie ve Martin Hellman, maliyeti 20 milyon dolar olan "DES-Crack" isimli DES algoritmasının anahtarını 1 günde bulabilecek bir makine önerdiler. 1993 yılında Michael Wiener, maliyeti 1 milyon dolar olan ve DES algoritmasının anahtarını 7 saatte bulabilecek bir makine önerdi. Her iki çalışma da yüksek maliyetten dolayı gerçekleşmedi. 1990 yılında Eli Bilham ve Adi Shamir diferensiyel kriptanaliz metodu ile DES üzerinde bir saldırı gerçekleştirdiler fakat bu girişim başarısız oldu. 1993 yılında Mitsuru Matsui DES algoritması üzerinde lineer kriptanaliz yöntemi tasarladı. Bu, DES üzerinde uygulanan ilk deneysel kriptanaliz yöntemi oldu. Daha sonra ise, DES algoritması için özel olarak Davies saldırısı yöntemi tasarlandı. Donald Davies tarafından önerilen bu saldırı, Eli Bilham ve Biryukov tarafından geliştirildi. Saldırı, $2^{50}$ hesaplama maliyetine sahipti. 1998 yılında EFF, 250 bin dolar maliyetle "Deep Crack" isminde, DES algoritmasının tüm anahtar ihtimallerini deneyen bir makine üretti. Saniyede 90 milyar anahtarı test edebilen ve her biri 64 mikroçip içeren 27 karttan oluşan bu makine, 56 bitlik bir anahtar ile şifrelenmiş bir metni 56 saatte kırdı. Bu olay, DES algoritmasına olan güvenin sarsılmasına yol açtı. Bu olaydan sonra 1999 yılında Triple DES (3DES) yöntemi geliştirildi.

\subsubsection{Encryption}

\begin{enumerate}
    \item Veri, sabit bir permütasyon tablosu (initial permutation) kullanılarak yeniden düzenlenir.
    \item Veri, Feistel yapısı ile 16 tur işlenir. Her turda, veri $L_i$ (sol) ve $R_i$ (sağ) olmak üzere 32-bitlik iki yarıya bölünür. Sağ taraf, bir fonksiyon ve bir alt anahtarla işlenir, ardından sol tarafla XOR yapılır. Sol ve sağ taraf yer değiştirir.
    \item İşlenen veri (final permutation), başlangıç permütasyonunun tersine göre düzenlenir.
    \item Şifreleme işlemi tamamlanır.
\end{enumerate}

\subsubsection{Python Kodu}

\begin{lstlisting}[language=Python]
from Crypto.Cipher import DES
from Crypto.Util.Padding import pad, unpad

def des_encrypt(plaintext, key):
    if len(key) != 8:
        raise ValueError("")

    cipher = DES.new(key, DES.MODE_ECB)
    padded_text = pad(plaintext.encode(), 8)
    encrypted = cipher.encrypt(padded_text)
    return encrypted

def des_decrypt(ciphertext, key):
    if len(key) != 8:
        raise ValueError("")

    cipher = DES.new(key, DES.MODE_ECB)
    decrypted_padded_text = cipher.decrypt(ciphertext)
    decrypted = unpad(decrypted_padded_text, 8)
    return decrypted.decode()

plaintext = "Hello, World!"
key = b"secretky"
encrypted = des_encrypt(plaintext, key)
decrypted = des_decrypt(encrypted, key)
print("Encrypted:", encrypted)
print("Decrypted:", decrypted)
\end{lstlisting}

\newpage

\subsection{3DES (Triple Data Encryption Standard)}

3DES, DES algoritmasının güvenlik açıklarını kapatmak için geliştirilmiş bir genişletilmiş versiyonudur. 3 farklı anahtar veya aynı anahtarın farklı kombinasyonlarını kullanarak 3 adımda şifreleme ve şifre çözme işlemleri gerçekleştirir. Bir veri bloğu üzerinde DES algoritması şu sırayla üç kez uygulanır:

\begin{enumerate}
    \item İlk anahtar ile DES şifrelemesi uygulanır.
    \item İkinci anahtar ile DES şifre çözmesi uygulanır.
    \item Üçüncü anahtar ile tekrar DES şifrelemesi uygulanır.
\end{enumerate}

\[ C = E_{K_3}(D_{K_2}(E_{K_1}(P)))\]

Burada, $P$, düz metin (plaintext); $C$, şifrelenmiş metin (ciphertext); $E$, DES şifreleme işlemi; $D$, DES şifre çözme işlemidir.

\subsubsection{Python Kodu}

\begin{lstlisting}[language=Python]
from Crypto.Cipher import DES3
from Crypto.Util.Padding import pad, unpad
from Crypto.Random import get_random_bytes

def triple_des_encrypt(plaintext, key):
    cipher = DES3.new(key, DES3.MODE_ECB)
    padded_text = pad(plaintext.encode(), 8)
    encrypted = cipher.encrypt(padded_text)
    return encrypted

def triple_des_decrypt(encrypted, key):
    cipher = DES3.new(key, DES3.MODE_ECB)
    decrypted_padded_text = cipher.decrypt(encrypted)
    decrypted = unpad(decrypted_padded_text, 8)
    return decrypted.decode()

key = get_random_bytes(24)
plaintext = "Hello, World!"
DES3.adjust_key_parity(key)
encrypted = triple_des_encrypt(plaintext, key)
decrypted = triple_des_decrypt(encrypted, key)
print("Encrypted:", encrypted)
print("Decrypted:", decrypted)
\end{lstlisting}

\newpage

\subsection{AES (Advanced Encryption Standard)}

AES (Advanced Encryption Standard), 2001 yılında Rijndael algoritması temel alınarak geliştirilmiş modern ve güvenli bir blok şifreleme algoritmasıdır. Veri güvenliği için dünya çapında yaygın olarak kullanılan bir standarttır. AES, sabit bir 128 bit blok boyutunda çalışır. Farklı güvenlik seviyeleri için 3 farklı anahtar uzunluğunu destekler: 128 bit (10 tur), 192 bit (12 tur) ve 256 bit (14 tur). AES, hem donanım hem de yazılım ortamlarında hızlıdır ve brute-force saldırılarına karşı oldukça dayanıklıdır.

\subsubsection{Encryption}

\begin{enumerate}
    \item Düz metin, ilk olarak anahtar ile XOR işlemine girer.
    \item Her tür şu 4 adımdan oluşur:
    \begin{itemize}
        \item \textbf{SubBytes (Bayt Değiştirme)}: Her bayt, S-box adı verilen bir tablo yardımıyla bir başkasıyla değiştirilir. Doğrusal olmayan bir işlemdir. Durum matrisi üzerinde uygulanır. 128 bitlik bir anahtar için durum matrixi her elemanı 8 bit olan 4x4'lük bir matristir. Bu matrisin her bir elemanı 16'lık sayı tabanında yazılır. Durum matrisindeki her eleman, elemanın satır-sütun koordinatı ile S kutusu üzerindeki o koordinatta bulunan değerle değiştirilir.
        \item \textbf{ShiftRows (Satır Kaydırma)}: Veri bloğundaki satırlar belirli bir düzene göre kaydırılır. Güncellenmiş durum matrisi üzerinde uygulanır. Her satır belirli bir oranda kaydırılır. Her n. satırda $n - 1$'lik bir kaydırma yapılır, yani ilk satırda herhangi bir kaydırma olmazken, diğer satırlarda kendisinin bir eksiği kadar kaydırma yapılır.
        \item \textbf{MixColumns (Sütun Karıştırma)}: Sütunlar matematiksel bir işlemle karıştırılır. Kaydırılan durum matrisi, sabit bir matris ile çarpılır.
        \item \textbf{AddRoundKey (Tur Anahtarı Ekleme)}: Veri bloğu, o tura ait alt anahtar ile XOR yapılır. Toplama işlemi sütun bazlı yapılır.
    \end{itemize}
    \item Son turda, SubBytes, ShiftRows ve AddRoundKey işlemleri yapılır. MixColumns atlanır.
    \item Tüm turlar tamamlandıktan sonra şifreli metin elde edilir.
\end{enumerate}

\subsubsection{Python Kodu}

\begin{lstlisting}[language=Python]
from Crypto.Cipher import AES
from Crypto.Util.Padding import pad, unpad
from Crypto.Random import get_random_bytes

def aes_encrypt(plaintext, key):
    cipher = AES.new(key, AES.MODE_CBC)
    iv = cipher.iv
    padded_text = pad(plaintext.encode(), AES.block_size)
    encrypted = cipher.encrypt(padded_text)
    return iv + encrypted

def aes_decrypt(encrypted, key):
    iv = encrypted[:AES.block_size]
    actual_encrypted_text = encrypted[AES.block_size:]
    cipher = AES.new(key, AES.MODE_CBC, iv)
    decrypted_padded_text = cipher.decrypt(actual_encrypted_text)
    decrypted = unpad(decrypted_padded_text, AES.block_size)
    return decrypted.decode()

key = get_random_bytes(24)
plaintext = "Hello, World!"
encrypted = triple_des_encrypt(plaintext, key)
decrypted = triple_des_decrypt(encrypted, key)
print("Encrypted:", encrypted)
print("Decrypted:", decrypted)
\end{lstlisting}

\newpage

\subsection{Salsa20}

2005 yılında Daniel J. Bernstein tarafından tasarlananmış bir akış şifreleme algoritmasıdır. Yüksek hızda ve düşük donanım gereksinimleriyle çalışması için geliştirilmiştir. Salsa20, sabit zamanlı algoritma olarak tasarlanmıştır, bu da yan kanal saldırılarına karşı dayanıklılık sağlar. Anahtar uzunluğu 256 bittir. Temel olarak, XOR, ekleme, döndürme ve karıştırma işlemlerine dayanır. Bir anahtar ve 64 bitlik nonce (keyfi sayı) kullanarak bir keystream oluşturur ve bu keystream ile veriyi XOR işlemini sokarak şifreler. Nonce, aynı anahtarla şifreleme yapılırken keystream'lerin farklı olmasını sağlar. Blok sayacı, bloklar arasında tekrar eden keystream olmasını önler.

\subsubsection{Encryption}

\begin{enumerate}
    \item 512 bitlik bir başlangıç durumu oluşturulur. Bu durum; sabit dizi, anahtar, nonce ve blok sayacından oluşur.
    \item 20 tur boyunca durum üzerinde XOR, quarterround (çeyrek tur), rowround (satır turu) ve columnround (sütun turu) işlemleri yapılır.
    \item Bu turlar sonucunda oluşan veri keystream olarak adlandırılır.
    \item Düz metin bu keystream ile XOR işlemine sokularak şifrelenir. Aynı keystream ile şifrelenmiş metin tekrar XOR işlemine sokularak düz metin elde edilir.
\end{enumerate}

\subsubsection{Python Kodu}

\begin{lstlisting}[language=Python]
from Crypto.Cipher import Salsa20

def salsa20_encrypt(plaintext, key):
    cipher = Salsa20.new(key=key)
    encrypted = cipher.encrypt(plaintext.encode())
    ciphertext = cipher.nonce + encrypted
    return ciphertext

def salsa20_decrypt(ciphertext, key):
    nonce = ciphertext[:8]
    cipher = Salsa20.new(key=key, nonce=nonce)
    plaintext = cipher.decrypt(ciphertext[8:]).decode()
    return plaintext

key = b"secretsecretkey!"
plaintext = "Hello, World!"
ciphertext = salsa20_encrypt(plaintext, key)
decrypted = salsa20_decrypt(ciphertext, key)
print("Encrypted:", ciphertext)
print("Decrypted:", decrypted)
\end{lstlisting}

\newpage

\subsection{Blowfish}

1993 yılında Bruce Schneier tarafından tasarlanmıştır. Blok boyutu 64 bittir, bu nedenle modern protokoller kadar güçlü değildir. 32 bit ile 448 bit arasında anahtar uzunluğuna sahip olabilir. Feistel ağına dayalıdır. Hem donanım hem de yazılımda hızlıdır. Patent veya lisans kısıtlaması yoktur.

\subsubsection{Encryption}

\begin{enumerate}
    \item Kullanıcı tarafından verilen anahtar, P-dizisi ve dört S-box'ı başlatır. Toplamda yaklaşık 4168 baytlık bir yapı kullanılır.
    \item Düz metin 64-bitlik iki parçaya bölünür: $L$ (sol) ve $R$ (sağ).
    \item $L$ ve $R$ parçaları 16 tur boyunca işlem görür. Her turda, $L$ ve $R$ bir P-değeri ve S-box'lar kullanılarak değiştirilir. 16 tur sonunda $L$ ve $R$ yer değiştirir ve birleştirilir. XOR işlemleriyle şifreli metin elde edilir.
\end{enumerate}

\subsubsection{Python Kodu}

\begin{lstlisting}[language=Python]
from Crypto.Cipher import Blowfish
from Crypto.Util.Padding import pad, unpad

def blowfish_encrypt(plaintext, key):
    cipher = Blowfish.new(key, Blowfish.MODE_CBC)
    iv = cipher.iv
    padded_text = pad(plaintext.encode(), Blowfish.block_size)
    ciphertext = cipher.encrypt(padded_text)
    return iv + ciphertext

def blowfish_decrypt(ciphertext, key):
    iv = ciphertext[:Blowfish.block_size]
    actual_ciphertext = ciphertext[Blowfish.block_size:]
    cipher = Blowfish.new(key, Blowfish.MODE_CBC, iv)
    padded_plaintext = cipher.decrypt(actual_ciphertext)
    plaintext = unpad(padded_plaintext, Blowfish.block_size)
    decrypted = plaintext.decode()
    return decrypted

key = b"secretsecretkey!"
plaintext = "Hello, World!"
ciphertext = blowfish_encrypt(plaintext, key)
decrypted = blowfish_decrypt(ciphertext, key)
print("Encrypted:", ciphertext)
print("Decrypted:", decrypted)
\end{lstlisting}

\newpage

\subsection{Twofish}

Bruce Schneier tarafından tasarlanmıştır. Blok boyutu 128 bittir. Anahtar uzunluğu 128, 192, 256 bit olabilir. Feistel yapısına dayanır. Tescil ya da lisans kısıtlaması yoktur. Twofish, Feistel ağı ve MD5 matrisleri kullanır.

\subsubsection{Encryption}

\begin{enumerate}
    \item Kullanıcının verdiği anahtar, anahtar programlama işlemi ile genişletilir. Subkey (alt anahtar) ve S-box tabloları oluşturulur.
    \item Düz metin, 16 baytlık bloklara ayrılır. Her turda, düz metin bloğuna alt anahtarlarla XOR işlemi uygulanır. Veriler, S-boxlar ile karıştırılır. Pseudo Hamart Transform (PHT) ile veriler arasındaki karışım artırılır. Veriler, bit seviyesinde döndürülür.
    \item Tüm şifreleme turları tamamlandıktan sonra son bir XOR işlemi yapılır ve şifreli metin elde edilir.
\end{enumerate}

\newpage

\subsection{ChaCha20}

Daniel J. Bernstein tarafından geliştirilmiştir. Hızlı, güvenli ve basit bir akış şifreleme algoritmasıdır. Salsa20 algoritmasının geliştirilmiş bir versiyonudur. Anahtar uzunluğu 256 bit (32 bayt). Nonce değeri, 96 bittir. Blok boyutu 512 bit (64 bayt). 

\subsubsection{Encryption}

\begin{enumerate}
    \item 256 bit uzunluğunda bir gizli anahtar (key) ve 96 bitlik bir nonce gerektirir. 32 bitlik bir sayaç ile çalışır.
    \item 4 sabit değer, 256 bitlik anahtar, 96 bit nonce değeri ve 32 bit sayaç birleştirilerek 4x4 boyutunda durum matrisi oluşturulur.
    \item Durum matrisi üzerinde bir dizi "Quarter Round" işlemi yapılır. Toplamda 20 tur round yapılır.
    \item Durum matrisi kullanılarak bir anahtar akışı (key stream) oluşturulur. Şifreleme ve şifre çözme için bu akış kullanılır.
    \item Anahtar akışı ile düz metin XOR işlemine girer böylece şifreli metin elde edilir. 
\end{enumerate}

\subsubsection{Python Kodu}

\begin{lstlisting}[language=Python]
from Crypto.Cipher import ChaCha20
from Crypto.Random import get_random_bytes

def chacha20_encrypt(plaintext, key):
    nonce = get_random_bytes(12)
    cipher = ChaCha20.new(key=key, nonce=nonce)
    ciphertext = cipher.encrypt(plaintext.encode())
    return nonce, ciphertext

def chacha20_decrypt(ciphertext, key, nonce):
    cipher = ChaCha20.new(key=key, nonce=nonce)
    plaintext = cipher.decrypt(ciphertext).decode()
    return plaintext

key = get_random_bytes(32)
plaintext = "Hello, World!"
nonce, ciphertext = chacha20_encrypt(plaintext, key)
decrypted = chacha20_decrypt(ciphertext, key, nonce)
print("Encrypted:", ciphertext)
print("Decrypted:", decrypted)
\end{lstlisting}

\newpage

\subsection{RC2 (Rivest Cipher 2)}

RC2, Ron Rivest tarafından 1987 yılında tasarlanmıştır. 64 bitlik blok boyutlarıyla çalışır. Anahtar boyutu 1-128 bit olabilir. ARC2 olarak da bilinir.

\subsubsection{Encrpytion}

\begin{enumerate}
    \item Anahtar türetme algoritması, kullanıcının sağladığı anahtarı belirli bir uzunluğa genişletir.
    \item Veri, 64 bitlik bloklara bölünür.
    \item Her blok üzerinde RC2'nin belirlediği dönüşümler uygulanır.
    \item Çıkış, şifrelenmiş veri bloklarıdır.
\end{enumerate}

\subsubsection{Python Kodu}

\begin{lstlisting}[language=Python]
from Crypto.Cipher import ARC2
from Crypto.Util.Padding import pad, unpad
from Crypto.Random import get_random_bytes

def rc2_encrypt(plaintext, key):
    cipher = ARC2.new(key, ARC2.MODE_CBC)
    ciphertext = cipher.encrypt(pad(plaintext.encode(), ARC2.block_size))
    return ciphertext, cipher.iv

def rc2_decrypt(ciphertext, key, iv):
    cipher = ARC2.new(key, ARC2.MODE_CBC, iv)
    plaintext = unpad(cipher.decrypt(ciphertext), ARC2.block_size)
    return plaintext.decode()

key = get_random_bytes(16)
plaintext = "Hello, World!"
ciphertext, iv = rc2_encrypt(plaintext, key)
decrypted = rc2_decrypt(ciphertext, key, iv)
print("Encrypted:", ciphertext)
print("Decrypted:", decrypted)
\end{lstlisting}

\newpage

\subsection{RC4 (Rivest Cipher 4)}

RC4, Ron Rivest tarafından 1987 yılında tasarlanmıştır. Artık güvenli kabul edilmemektedir.

\subsubsection{Encrpytion}

\begin{enumerate}
    \item Bir KSA (Key Scheduling Algorithm) ile, bir anahtar kullanılarak 256 baytlık bir S-box oluşturulur. S-box, anahtarın başlangıç durumunu temsil eder. S-box, 0'dan 255'e kadar tüm bayt değerlerini içerir.
    \item Pseudo-random Generation Algorithm (PRGA) ile S-box kullanılarak bir rastgele bayt akışı üretilir. Bu rastgele bayt akışı, açık metinle XOR işlemine girerek şifreli metnini üretir.
\end{enumerate}

\subsubsection{Python Kodu}

\begin{lstlisting}[language=Python]
def ksa(key):
    S = list(range(256))
    j = 0
    for i in range(256):
        j = (j + S[i] + key[i % len(key)]) % 256
        S[i], S[j] = S[j], S[i]
    return S

def prga(S):
    i = 0
    j = 0
    while True:
        i = (i + 1) % 256
        j = (j + S[i]) % 256
        S[i], S[j] = S[j], S[i]
        yield S[(S[i] + S[j]) % 256]

def rc4_encrypt_decrypt(message, key):
    key = [ord(c) for c in key]
    S = ksa(key)
    keystream = prga(S)
    result = bytes([message[i] ^ next(keystream) for i in range(len(message))])
    return result

key = "secretkey"
plaintext = "Hello, World!"
ciphertext = rc4_encrypt_decrypt(plaintext.encode(), key)
decrypted = rc4_encrypt_decrypt(ciphertext, key)
print("Encrypted:", ciphertext)
print("Decrypted:", decrypted.decode())
\end{lstlisting}

\newpage